\phantomsection
\chapter[Training a Computer to Classify White Blood Cells]{Training a Computer to Classify White Blood Cells\chapsubhead{Phillip Compeau}}
\label{chapter:white_blood_cells}
\renewcommand{\chaptertitle}{Training a Computer to Classify White Blood Cells}
\addcontentsline{cc}{chapter}{Chapter \thechapter} % Adds chapter number to table of contents


%\clearpage \null
%\thispagestyle{empty}
%\AddToShipoutPictureBG*{% Add picture to current page
%  \AtStockLowerLeft{% Add picture to lower-left corner of paper stock
%    \includegraphics[width=\stockwidth]{images/CMYK_ed3/cover/alignment_final}}
%}
%\addtocounter{page}{-1}
%\clearpage

\FloatBarrier

\section{Introduction: How Are Blood Cells Counted?}
\label{sec:introduction}
\phantomsection

Your doctor sometimes counts your blood cells to ensure that they are within healthy ranges as part of a complete blood count. Blood cells consist of \textdef{red blood cells (RBCs)}{red blood cells (RBCs)}{FILL IN}, which transport oxygen via the hemoglobin protein, and \textdef{white blood cells (WBCs)}{white blood cells (WBCs)}{FILL IN}, which help identify and attack foreign cells as part of your immune system.

The classic device used for counting blood cells is the \textdef{hemocytometer}{hemocytometer}{FILL IN}. In this device, a technician filters a small amount of blood onto a gridded slide and then counts the number of cells of each type in the squares on the grid. As a result, the technician can estimate the number of each type of cell per volume of blood.

\texttt{NEED FIGURE HERE -- TO REPLACE VIDEO}

\begin{qbox}[%
Why might the size of the blood sample influence the estimate of blood cell count?
]\end{qbox}

If the hemocytometer seems old-fashioned, you are not wrong; it was invented by Louis-Charles Malassez 150 years ago. To reduce the human error inherent in using this device, what if we train a computer to count blood cells for us?

We will focus specifically on WBCs. A low WBC count may indicate a host of diseases that leave the immune system susceptible to attack; a high WBC count may indicate that an infection is present, or that a disease like leukemia has caused overproduction of WBCs.

WBCs divide into families based on their structure and function, and some diseases may cause an abnormally low or high count of a specific WBC classes. We therefore wish not only to identify WBCs in cellular images but also to \textdef{classify}{classify}{FILL IN} WBCs into their appropriate types.

We will work with a publicly available dataset (hosted at \url{https://github.com/Shenggan/BCCD_Dataset}) containing blood cell images that depict both RBCs and WBCs, as shown in \autoref{fig:three_families}. The cells have been applied with a stain in which a red-orange dye bonds to hemoglobin and a blue dye bonds to DNA and RNA. The red-orange dye will make the RBCs (which do not have DNA or RNA) look red, and the blue dye will make the WBC nuclei look purple.

\autoref{fig:three_families} also illustrates the three main families of WBCs: \textdef{granulocytes}{granulocytes}{FILL IN}, \textdef{lymphocytes}{lymphocytes}{FILL IN}, and \textdef{monocytes}{monocytes}{}.  Granulocytes have a \textdef{multilobular nucleus}{multilobular nucleus}{FILL IN}, which consists of several round “lobes” that are linked by thin strands of nuclear material. Monocyte and lymphocyte nuclei only have a single lobe, but the shapes of the nuclei are quite different: lymphocyte nuclei tend to have a more rounded shape (taking up a greater fraction of the cell’s volume), whereas monocyte nuclei have a more irregular shape.

\begin{figure}[p]
\centering
\tabcolsep = 2.5em
\mySfFamily
\begin{tabular}{c c c}
\end{tabular}
\caption{Three images from the blood cell image dataset showing three types of WBCs. In our dataset, these cells correspond to image IDs 3, 15, and 20. (Left) A specific subtype of granulocyte called a neutrophil, illustrating the multilobular structure of this WBC family. (Center) A monocyte with a single, irregularly-shaped nucleus. (Right) A lymphocyte with a small, round nucleus.}
\label{fig:alignment_game}
\end{figure}

In this module, our goal is twofold. First, can we excise, or \textdef{segment}{segment}{FILL IN}, WBCs from these cellular images? Second, can we train a computer to classify WBCs by family? To perform these tasks, we will enlist CellOrganizer (\url{http://www.cellorganizer.org}), a powerful software resource that can perform a wide variety of automated analyses on cellular images.

When you look at the cells in the figure above, you may think that our two tasks will be easy. Identifying WBCs only requires identifying the large purplish regions, and classifying them is just a matter of categorizing them according to the differences in shape that we described above. Yet even after decades of research into computer vision, researchers struggle to attain the precision of the human eye.


\begin{figure*}[b]
\flushright
\includegraphics[width = 0.129\stockwidth]{images/CMYK_ed3/cover/trex}
\end{figure*}

\phantom{x}